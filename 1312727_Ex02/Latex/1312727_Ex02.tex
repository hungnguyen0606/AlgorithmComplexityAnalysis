\documentclass{article}
\usepackage[utf8]{vietnam}

\usepackage{amsmath}
\usepackage{amsthm}
\usepackage{amsfonts}
\usepackage{mathtools}
\DeclarePairedDelimiter{\ceil}{ \lceil}{ \rceil}

%----------------------------------------
\usepackage{listings}
\usepackage{titlesec}
\setcounter{secnumdepth}{4}
\usepackage{multirow}
\usepackage{float}
%---------------------------------------


\usepackage{graphicx}
\graphicspath{{Images/}}
\usepackage{pgfplots}

\usepgfplotslibrary{external}

\tikzexternalize
\usepackage{xcolor}
\usepackage{listings}
\usepackage{caption}

\usepackage{color}


\definecolor{mycomment}{rgb}{0,0.6,0}
\definecolor{codegray}{rgb}{0.5,0.5,0.5}
\definecolor{codepurple}{rgb}{1,1,0}
\definecolor{backcolour}{rgb}{1,1,1}
\definecolor{source}{rgb}{1,1,1} 
\lstdefinestyle{C++}{
    backgroundcolor=\color{white},   
    commentstyle=\color{gray},
    keywordstyle=\color{blue},
    numberstyle=\tiny\color{black},
    stringstyle=\color{purple},
    basicstyle=\footnotesize,
    breakatwhitespace=false,         
    breaklines=true,                 
    captionpos=b,                    
    keepspaces=true,                 
    numbers=left,                    
    numbersep=5pt,                  
    showspaces=false,                
    showstringspaces=false,
    showtabs=false,                  
    tabsize=2,
		frame = single,
		language = c++,
		extendedchars=true
		%encoding = utf8
}


\lstset{%
	literate=
	{đ}{{\dj}}1
{â}{{\^a}}1
{ă}{{\u{a}}}1
{ê}{{\^e}}1
{ô}{{\^o}}1
{ơ}{{\ohorn}}1
{ư}{{\uhorn}}1
{á}{{\'a}}1
{à}{{\`a}}1
{ả}{\h{a}}1
{ã}{{\~a}}1
{ạ}{\textsubdot{a}}1
{ấ}{\'{\^a}}1
{ầ}{\`{\^a}}1
{ẩ}{\h{\^a}}1
{ẫ}{\~{\^a}}1
{ậ}{\textsubdot{\^a}}1
{ắ}{\'{\u{a}}}1
{ằ}{\`{\u{a}}}1
{ẳ}{\h{\u{a}}}1
{ẵ}{\~{\u{a}}}1
{ặ}{\textsubdot{\u{a}}}1
{é}{{\'e}}1
{è}{{\`e}}1
{ẻ}{\h{e}}1
{ẽ}{{\~e}}1
{ẹ}{\textsubdot{e}}1
{ế}{\'{\^e}}1
{ề}{\`{\^e}}1
{ể}{\h{\^e}}1
{ễ}{\~{\^e}}1
{ệ}{\textsubdot{\^{e}}}1
{í}{{\'i}}1
{ì}{{\`i}}1
{ỉ}{\h{i}}1
{ĩ}{{\~i}}1
{ị}{\textsubdot{i}}1
{ó}{{\'o}}1
{ò}{{\`o}}1
{ỏ}{\h{o}}1
{õ}{{\~o}}1
{ọ}{\textsubdot{o}}1
{ố}{\'{\^o}}1
{ồ}{\`{\^o}}1
{ổ}{\h{\^o}}1
{ỗ}{\~{\^o}}1
{ộ}{\textsubdot{\^o}}1
{ớ}{\'{\ohorn}}1
{ờ}{\`{\ohorn}}1
{ở}{\h{\ohorn}}1
{ỡ}{\~{\ohorn}}1
{ợ}{\textsubdot{\ohorn}}1
{ú}{{\'u}}1
{ù}{{\`u}}1
{ủ}{\h{u}}1
{ũ}{{\~u}}1
{ụ}{\textsubdot{u}}1
{ứ}{\'{\uhorn}}1
{ừ}{\`{\uhorn}}1
{ử}{\h{\uhorn}}1
{ữ}{\~{\uhorn}}1
{ự}{\textsubdot{\uhorn}}1
{ý}{{\'y}}1
{ỳ}{{\`y}}1
{ỷ}{\h{y}}1
{ỹ}{{\~y}}1
{ỵ}{\textsubdot{y}}1
{Đ}{{\DJ}}1
{Â}{{\^A}}1
{Ă}{{\u{A}}}1
{Ê}{{\^E}}1
{Ô}{{\^O}}1
{Ơ}{{\OHORN}}1
{Ư}{{\UHORN}}1
{Á}{{\'A}}1
{À}{{\`A}}1
{Ả}{\h{A}}1
{Ã}{{\~A}}1
{Ạ}{\textsubdot{A}}1
{Ấ}{\'{\^A}}1
{Ầ}{\`{\^A}}1
{Ẩ}{\h{\^A}}1
{Ẫ}{\~{\^A}}1
{Ậ}{\textsubdot{\^A}}1
{Ắ}{\'{\u{A}}}1
{Ằ}{\`{\u{A}}}1
{Ẳ}{\h{\u{A}}}1
{Ẵ}{\~{\u{A}}}1
{Ặ}{\textsubdot{\u{A}}}1
{É}{{\'E}}1
{È}{{\`E}}1
{Ẻ}{\h{E}}1
{Ẽ}{{\~E}}1
{Ẹ}{\textsubdot{E}}1
{Ế}{\'{\^E}}1
{Ề}{\`{\^E}}1
{Ể}{\h{\^E}}1
{Ễ}{\~{\^E}}1
{Ệ}{\textsubdot{\^{E}}}1
{Í}{{\'I}}1
{Ì}{{\`I}}1
{Ỉ}{\h{I}}1
{Ĩ}{{\~I}}1
{Ị}{\textsubdot{I}}1
{Ó}{{\'O}}1
{Ò}{{\`O}}1
{Ỏ}{\h{O}}1
{Õ}{{\~O}}1
{Ọ}{\textsubdot{O}}1
{Ố}{\'{\^O}}1
{Ồ}{\`{\^O}}1
{Ổ}{\h{\^O}}1
{Ỗ}{\~{\^O}}1
{Ộ}{\textsubdot{\^O}}1
{Ớ}{\'{\OHORN}}1
{Ờ}{\`{\OHORN}}1
{Ở}{\h{\OHORN}}1
{Ỡ}{\~{\OHORN}}1
{Ợ}{\textsubdot{\OHORN}}1
{Ú}{{\'U}}1
{Ù}{{\`U}}1
{Ủ}{\h{U}}1
{Ũ}{{\~U}}1
{Ụ}{\textsubdot{U}}1
{Ứ}{\'{\UHORN}}1
{Ừ}{\`{\UHORN}}1
{Ử}{\h{\UHORN}}1
{Ữ}{\~{\UHORN}}1
{Ự}{\textsubdot{\UHORN}}1
{Ý}{{\'Y}}1
{Ỳ}{{\`Y}}1
{Ỷ}{\h{Y}}1
{Ỹ}{{\~Y}}1
{Ỵ}{\textsubdot{Y}}1
}






\begin{document}

\begin{titlepage}
\begin{center}
	\large{\textbf{ĐẠI HỌC QUỐC GIA THÀNH PHỐ HỒ CHÍ MINH}}
	\large{\textbf{TRƯỜNG ĐẠI HỌC KHOA HỌC TỰ NHIÊN}}
	\large{\textbf{KHOA CÔNG NGHỆ THÔNG TIN}}
	
	\begin{figure}[H]
		\centerline{\includegraphics[scale = 0.5]{logo}}
	\end{figure}

	\Large{\textbf{Nhập môn phân tích độ phức tạp thuật toán}}
	\Large{\textbf{BT01: Phép gán và phép so sánh}}

\end{center}
	\vfill
\begin{flushright}
	
	\begin{tabular}{l l l}
		GVLT: &Thầy Trần Đan Thư\\
		&\\
		GVTH: &Thầy Nguyễn Đức Thân\\
		&Thầy Trương Toàn Thịnh\\
		&Thầy Nguyễn Vinh Tiệp\\
		&Thầy Nguyễn Sơn Hoàng Quốc\\
		&\\
		Sv: &Nguyễn Phan Mạnh Hùng & 1312727\\
	\end{tabular}
\end{flushright}

\vfill
\end{titlepage}
\pagebreak
\thispagestyle{empty}
\tableofcontents{}
\pagebreak
%------------------------------------------------------------
\pagenumbering{arabic}


\section{Quy ước}
	\begin{flushleft}
		$A(n)$: số lượng phép gán.\\
		$C(n)$: số lượng phép so sánh.
	\end{flushleft}


\section{Mã nguồn}
\lstinputlisting[style = C++]{Sources/Sum.cpp}


\section{Nhận xét - Dự đoán độ phức tạp thuật toán}
	\begin{figure}[H]
		\includegraphics[scale = 0.5]{Calculate}
	\end{figure}
	\begin{flushleft}
		Ta nhận thấy vòng lặp trong cùng chỉ thực hiện khi $j \leq i*i$ hay:
		\begin{equation}\label{eq:Eq1} 
			n - i \leq i*i
		\end{equation}
		Giải bất phương \ref{eq:Eq2} với ẩn $i \geq 0$, tham số $n$ ta được:
		\begin{equation}\label{eq:Eq2}
			i \geq  \left\lceil\frac{-1 +\sqrt{4n + 1}}{2}\right\rceil = \alpha
		\end{equation}
		Ta tính được:
		\begin{equation}\label{eq:Eq3}
			C(n) = n + \alpha + \Sigma_{i = \alpha}^{n}\left(i^2+i-n+2\right)
		\end{equation}
		Mà $\forall n \geq 2,~alpha \leq \frac{n}{2}$, nên:\\
		\begin{gather}
			\begin{align*}
			C(n) & \geq n + \alpha + \Sigma_{i = \left\lceil \frac{n}{2} \right\rceil + 1}^{n}\left(i^2+i-n+2\right)\\
			& \geq n + \Sigma_{i = \left\lceil \frac{n}{2} \right\rceil + 1}^{n}\left( \frac{n}{2} \right)^2 + \frac{n}{2} - n \\
			& \geq n + \left( \frac{n}{2} - 3 \right) \times \left[ \left( \frac{n}{2} \right)^2  - n \right] = \frac{n^3}{8} - \frac{5n^2}{4} + 4\times n\\
			\Rightarrow 8 \times C(n) &\geq n\times \left(  n^3 - 10n^2 + 32 \right) \geq n^3 
			\end{align*}
		\end{gather}
		Do đó: 
		\begin{equation}\label{eq:Eq4}
			C(n) \in \Omega(n^3)
		\end{equation}
		Bên cạnh đó, $\forall n \geq 2$
		\begin{gather}
			\begin{align*}
				C(n) &\leq n + \Sigma_{i = \left\lceil \frac{n}{2} \right\rceil + 1}^{n}\left(n^2 + n - n\right) \\
				&\leq n + \left(\frac{n}{2}\right) \times n^2 = n + \frac{n^3}{2} \leq 10n^3
			\end{align*}
		\end{gather}	
		nên 
		\begin{equation}\label{eq:Eq5}
			C(n) \in O(n^3)
		\end{equation}
		Từ \ref{eq:Eq4} và \ref{eq:Eq5}, ta được\\
		\begin{equation}\label{eq:Eq6}
			C(n) \in \Theta(n^3)
		\end{equation}
		
		Tuy vậy, ta có thể tính chính xác số phép so sánh bằng cách khai triển \ref{eq:Eq3}, ta được:
		\begin{equation}
			C(n) = 2n^3 + n\times(16+6\alpha) - 2\alpha^3 -4\alpha + 12
		\end{equation}
		Với công thức trên ta dễ dàng chứng minh được $C(n) \in \Theta(n^3)$ mà không thông qua chứng minh bất đẳng thức ở trên.\\
		Đối với phép gán, ta chứng minh tương tự phép so sánh sẽ cho ra độ phức tạp như trên.
	\end{flushleft}
\end{document}